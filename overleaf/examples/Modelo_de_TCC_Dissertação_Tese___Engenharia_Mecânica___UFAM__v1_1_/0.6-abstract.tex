\singlespacing %\doublespacing %

\noindent \CITACAOAUTOR. \textbf{\TITLE. \ANO. \pageref{LastPage}f.} Undergraduate thesis (Mechanical Engineering) - Federal University of Amazonas, \LOCAL, \ANOD.  

  
%\noindent DE TAL, Fulano. \textbf{Título do Trabalho. Ano da entrega. número de páginas f.} Trabalho de Conclusão de Curso (Engenharia Mecânica) - Universidade Federal do Amazonas, Manaus, ano da defesa.  
\vspace{0.5 cm}

\onehalfspacing %
\noindent Write your text here. Write your text here. Write your text here. Write your text here. Write your text here. Write your text here. Write your text here. Write your text here. Write your text here. Write your text here. Write your text here. Write your text here. Write your text here. Write your text here. Write your text here. Write your text here. Write your text here. 

\noindent \textbf{Keywords:} Keyword 1, Keyword 2, Keyword 3, Keyword 4, Keyword 5.

%\noindent Efficient diagnosis of emissions from combustion processes plays a key role in their control, an essential part of the overall effort to mitigate the increasing greenhouse effect. In industrial furnaces, a set of sensors ($CO_x$, $SO_x$, $NO_x$) at the exhaust is used to monitor pollutant rates, thus providing the necessary information for control purposes. In the case of natural gas furnaces, measurements of $O_2$ and $CO_2$ contents are used to check the condition of the combustion process. In this work, we develop a soft sensor to estimate the $O_2$ and $CO_2$ contents at the exhaust of a natural gas prototype furnace from images of flames grabbed by a charge-coupled device (CCD) camera. Feature vectors obtained from computer processing of the grabbed images are used as input data to identify autoregressive moving average (ARMAX) ``black box'' models having $CO_2$ content as output. Estimates of $O_2$ content by a Kalman filter running a preliminary ARMAX model helps the overall performance of the soft sensor. Results show that the flame dynamics identified model is capable of yielding statistically significant estimates of both $O_2$ and $CO_2$ composition in the flue gas up to $10\,s$ before the arrival of actual $O_2$ measurements. This outcome suggests that the inclusion of the proposed soft sensor in the closed-loop control strategy of similar combustion processes might be advantageous.