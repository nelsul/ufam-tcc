\singlespacing

\noindent \CITACAOAUTOR. \textbf{\TITULO: \SUBTITULO. \ANO. \pageref{LastPage}f.} Trabalho de Conclusão de Curso (Engenharia de Software) - Universidade Federal do Amazonas, \LOCAL, \ANOD.

\vspace{0.5 cm}

\onehalfspacing

\noindent A crescente demanda por suporte à saúde mental no ensino superior impulsiona a busca por ferramentas tecnológicas que otimizem a gestão dos atendimentos. Visando aprimorar os serviços psicopedagógicos oferecidos na Universidade Federal do Amazonas (UFAM), este trabalho apresenta o desenvolvimento do IcompCare, um Sistema de Apoio Acadêmico. A plataforma web foi projetada para agregar eficiência, segurança e organização ao processo, servindo como uma evolução aos fluxos de trabalho tradicionais. A solução introduz um portal de agendamento online, onde alunos podem visualizar a disponibilidade da equipe e solicitar horários. Para os profissionais, o sistema oferece um painel de gestão de consultas e um prontuário eletrônico com dados criptografados. O sistema permite ainda o compartilhamento seguro de observações não-clínicas com professores, promovendo um acompanhamento integrado. O projeto utilizou .NET 8, Vue.js e PostgreSQL.

\noindent \textbf{Palavras-chave:} apoio psicológico, sistema web, gestão acadêmica, privacidade de dados, engenharia de software.

