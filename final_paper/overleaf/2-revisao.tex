\chapter{Referencial Teórico}
\label{cap:referencial}

Este capítulo apresenta a fundamentação teórica necessária para o desenvolvimento do Sistema de Apoio Acadêmico. São abordados os conceitos relacionados à gestão da saúde mental no ambiente universitário, seguidos pelos princípios de Engenharia de Software utilizados, incluindo arquitetura de sistemas, desenvolvimento web moderno e segurança da informação com foco na legislação vigente.

\section{Saúde Mental no Ensino Superior}

A saúde mental no contexto universitário tem se tornado um tema de crescente relevância. O ambiente acadêmico, caracterizado por prazos rigorosos, alta carga cognitiva e pressão por desempenho, pode atuar como um estressor significativo. Conforme apontado por \citeonline{leao2018}, a incidência de ansiedade e depressão entre universitários tende a ser superior à da população geral, o que demanda das instituições de ensino a criação de mecanismos de suporte eficazes.

A informatização desses serviços de apoio não é apenas uma questão de modernização, mas de acessibilidade e eficácia. Sistemas de gestão clínica permitem um acompanhamento mais próximo, reduzem o estigma associado à busca presencial por informações e garantem a continuidade do tratamento através de um histórico organizado e seguro.

\section{Arquitetura de Software}

A arquitetura de software refere-se à estrutura fundamental de um sistema, compreendendo seus componentes, as relações entre eles e os princípios que guiam seu design e evolução. Uma arquitetura bem definida é crucial para a manutenibilidade e escalabilidade do software.

\subsection{Clean Architecture}

A Clean Architecture (Arquitetura Limpa), proposta por \citeonline{martin2017}, é um padrão de design de software que visa a separação de preocupações. Seu principal objetivo é tornar o sistema independente de frameworks, banco de dados e interfaces de usuário.

O modelo organiza o software em camadas concêntricas, onde a regra de dependência estabelece que as camadas internas (Domínio e Aplicação) não devem conhecer nada sobre as camadas externas (Interface e Infraestrutura). Isso permite que as regras de negócio permaneçam puras e testáveis, enquanto detalhes técnicos podem ser substituídos com mínimo impacto no núcleo do sistema. Essa abordagem facilita a manutenção a longo prazo e a adaptação a novas tecnologias.

\subsection{Padrão API REST}

O estilo arquitetural REST (Representational State Transfer) define um conjunto de restrições para a criação de web services. Em uma API REST, a comunicação entre cliente e servidor ocorre através de requisições HTTP padronizadas, tratando os dados como recursos que podem ser criados, lidos, atualizados ou excluídos (operações CRUD).

A adoção de APIs REST permite o desacoplamento total entre o frontend e o backend. O servidor processa as regras de negócio e retorna dados (geralmente em formato JSON), enquanto o cliente (seja web ou mobile) consome esses dados e os apresenta ao usuário.

\section{Desenvolvimento Web}

O desenvolvimento de aplicações web evoluiu do modelo tradicional de renderização no servidor para abordagens focadas na experiência do usuário e na interatividade.

\subsection{Single Page Applications (SPA)}

Uma Single Page Application (SPA) é uma aplicação web que carrega uma única página HTML e atualiza dinamicamente o conteúdo conforme o usuário interage com o sistema, sem a necessidade de recarregar a página inteira a cada ação.

Essa abordagem, viabilizada por frameworks JavaScript modernos como Vue.js, React e Angular, proporciona uma experiência de usuário mais fluida e responsiva, semelhante à de aplicativos desktop nativos. O processamento da interface é transferido para o navegador do cliente, reduzindo a carga no servidor e o tráfego de rede.

\section{Segurança e Proteção de Dados}

Com a digitalização de registros médicos e acadêmicos, a segurança da informação torna-se um requisito não funcional crítico, envolvendo a confidencialidade, integridade e disponibilidade dos dados.

\subsection{Criptografia e Hash}

A criptografia é a técnica fundamental para garantir a confidencialidade dos dados. Ela consiste em codificar a informação de modo que apenas partes autorizadas, detentoras de uma chave específica, possam decifrá-la.

Para o armazenamento de senhas, utilizam-se funções de hash criptográfico. Diferente da criptografia simétrica ou assimétrica, o hash é uma via de mão única: ele transforma a senha em uma cadeia de caracteres de tamanho fixo impossível de ser revertida para o texto original. Isso garante que, mesmo em caso de vazamento do banco de dados, as senhas dos usuários não sejam expostas.