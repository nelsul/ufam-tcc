\chapter{Resultados e Discussões}
\label{chap:resultados}

Este capítulo apresenta os resultados obtidos com o desenvolvimento do Sistema de Apoio Acadêmico, demonstrando as principais funcionalidades implementadas e como elas atendem aos requisitos definidos. Serão exibidas as interfaces do sistema e discutidos os benefícios trazidos pela automação dos processos de agendamento e acompanhamento de estudantes.

\section{Apresentação do Sistema}

O sistema foi implantado e validado em ambiente de homologação, permitindo testar todos os fluxos de uso previstos. A interface foi projetada para ser intuitiva e responsiva, garantindo acessibilidade em diferentes dispositivos.

\subsection{Autenticação e Controle de Acesso}

A tela de login (Figura \ref{fig:login}) é o ponto de entrada para os usuários autenticados (profissionais, professores e administradores). O sistema utiliza autenticação via token JWT, garantindo segurança na comunicação.

A segurança dos dados foi um resultado crítico alcançado. Todas as senhas são armazenadas com hash criptográfico (Figura \ref{fig:hash}), garantindo que mesmo em caso de acesso indevido ao banco de dados, as credenciais dos usuários permaneçam protegidas.

\begin{figure}[H]
    \caption{\label{fig:login}Tela de Login do Sistema}
    \begin{center}
        \includegraphics[width=0.5\textwidth]{Imagens/login-page.png}
    \end{center}
    \legend{O Autor (2025)}
\end{figure}

\begin{figure}[H]
    \caption{\label{fig:hash}Armazenamento de Senhas com Hash Criptográfico}
    \begin{center}
        \includegraphics[width=0.6\textwidth]{Imagens/password-hash.png}
    \end{center}
    \legend{O Autor (2025)}
\end{figure}

\subsection{Fluxo de Agendamento Público}

Uma das principais funcionalidades do sistema é permitir que qualquer usuário visualize a disponibilidade dos profissionais e solicite um agendamento. A Figura \ref{fig:calendar} mostra a visualização do calendário público, onde é possível preencher e solicitar um atendimento.

\begin{figure}[H]
    \caption{\label{fig:calendar}Formulário de Agendamento Público}
    \begin{center}
        \includegraphics[width=0.4\textwidth]{Imagens/schedule-p1.png}
    \end{center}
    \legend{O Autor (2025)}
\end{figure}

Após preencher os dados e selecionar um horário, é mostrado o resumo do agendamento (Figura \ref{fig:request}).

\begin{figure}[H]
    \caption{\label{fig:request}Resumo do Agendamento}
    \begin{center}
        \includegraphics[width=0.35\textwidth]{Imagens/schedule-p2.png}
    \end{center}
    \legend{O Autor (2025)}
\end{figure}

Após a revisão do formulário, o sistema valida os dados e envia a solicitação (Figura \ref{fig:request-complete}).

\begin{figure}[H]
    \caption{\label{fig:request-complete}Confirmação da Solicitação de Agendamento}
    \begin{center}
        \includegraphics[width=0.7\textwidth]{Imagens/schedule-p3.png}
    \end{center}
    \legend{O Autor (2025)}
\end{figure}

O sistema dispara automaticamente um email de confirmação de recebimento para o solicitante (Figura \ref{fig:email-confirmation}), garantindo feedback imediato sobre a ação realizada.

\begin{figure}[H]
    \caption{\label{fig:email-confirmation}Email de Confirmação de Agendamento}
    \begin{center}
        \includegraphics[width=0.6\textwidth]{Imagens/schedule-p4-email.png}
    \end{center}
    \legend{O Autor (2025)}
\end{figure}

\subsection{Gestão de Agendamentos pelo Profissional}

O profissional possui um painel administrativo onde pode visualizar todas as solicitações recebidas (Figura \ref{fig:dashboard}). O sistema organiza os agendamentos por status (pendente, confirmado, cancelado), facilitando o gerenciamento diário.

\begin{figure}[H]
    \caption{\label{fig:dashboard}Painel de Gestão de Agendamentos}
    \begin{center}
        \includegraphics[width=0.8\textwidth]{Imagens/appointments-01.png}
    \end{center}
    \legend{O Autor (2025)}
\end{figure}

A partir deste painel, o profissional pode visualizar detalhes de cada agendamento (Figura \ref{fig:appointment-details}) e tomar ações como confirmar, cancelar ou remarcar.

\begin{figure}[H]
    \caption{\label{fig:appointment-details}Detalhes de um Agendamento}
    \begin{center}
        \includegraphics[width=0.7\textwidth]{Imagens/appointments-02.png}
    \end{center}
    \legend{O Autor (2025)}
\end{figure}

O sistema permite cadastrar um novo aluno com base nas informações do agendamento, ou mostra que o aluno já existe

\begin{figure}[H]
    \caption{\label{fig:appointment-actions}Vincular Estudante ao Agendamento}
    \begin{center}
        \includegraphics[width=0.3\textwidth]{Imagens/appointments-03.png}
    \end{center}
    \legend{O Autor (2025)}
\end{figure}

\begin{figure}[H]
    \caption{\label{fig:appointment-confirm}Confirmação de Vinculação do Estudante ao Agendamento}
    \begin{center}
        \includegraphics[width=0.4\textwidth]{Imagens/appointments-04.png}
    \end{center}
    \legend{O Autor (2025)}
\end{figure}

Em seguida, é necessário escolher o tipo de atendimento (visível apenas para o profissional) que implica na duração da sessão.

\begin{figure}[H]
    \caption{\label{fig:appointment-success}Seleção do Tipo de Atendimento}
    \begin{center}
        \includegraphics[width=0.4\textwidth]{Imagens/appointments-05.png}
    \end{center}
    \legend{O Autor (2025)}
\end{figure}

Após a confirmação pelo profissional, um email é enviado ao estudante contendo todos os detalhes do agendamento (Figura \ref{fig:appointment-email}), incluindo data, horário e informações sobre o atendimento.

\begin{figure}[H]
    \caption{\label{fig:appointment-email}Email de Confirmação de Agendamento}
    \begin{center}
        \includegraphics[width=0.4\textwidth]{Imagens/appointments-06.png}
    \end{center}
    \legend{O Autor (2025)}
\end{figure}

Caso seja necessário cancelar, o sistema também envia notificação apropriada ao estudante (Figura \ref{fig:appointment-cancel}).

\begin{figure}[H]
    \caption{\label{fig:appointment-cancel}Email de Cancelamento de Agendamento}
    \begin{center}
        \includegraphics[width=0.4\textwidth]{Imagens/appointments-07.png}
    \end{center}
    \legend{O Autor (2025)}
\end{figure}

\subsection{Realização de Sessões e Prontuário Eletrônico}

Para os atendimentos confirmados, o sistema oferece uma interface dedicada para a realização da sessão, onde é mostrado a atual sessão ativa e sessões anteriores (Figura \ref{fig:session}).
\begin{figure}[H]
    \caption{\label{fig:session}Tela de Sessões do Profissional}
    \begin{center}
        \includegraphics[width=0.8\textwidth]{Imagens/sessions-01.png}
    \end{center}
    \legend{O Autor (2025)}
\end{figure}

Durante a sessão, o profissional pode escrever o Prontuário Eletrônico utilizando um editor de texto rico (Figura \ref{fig:session-editor}) para organizar melhor suas anotações.

\begin{figure}[H]
    \caption{\label{fig:session-editor}Prontuário Eletrônico com Editor de Texto Rico}
    \begin{center}
        \includegraphics[width=0.8\textwidth]{Imagens/sessions-02.png}
    \end{center}
    \legend{O Autor (2025)}
\end{figure}

Ao finalizar a sessão (Figura \ref{fig:session-save}), é possível adicionar uma nota específica para validação futura.

\begin{figure}[H]
    \caption{\label{fig:session-save}Encerramento da Sessão}
    \begin{center}
        \includegraphics[width=0.3\textwidth]{Imagens/sessions-04.png}
    \end{center}
    \legend{O Autor (2025)}
\end{figure}

O histórico de sessões anteriores pode ser consultado (Figura \ref{fig:session-history}), permitindo ao profissional acompanhar a evolução do estudante ao longo do tempo. O acesso aos detalhes de sessões passadas permite uma continuidade efetiva no acompanhamento, com todas as anotações anteriores disponíveis de forma organizada e segura.

\begin{figure}[H]
    \caption{\label{fig:session-history}Histórico de Sessões do Estudante}
    \begin{center}
        \includegraphics[width=0.45\textwidth]{Imagens/sessions-05.png}
    \end{center}
    \legend{O Autor (2025)}
\end{figure}

\begin{figure}[H]
    \caption{\label{fig:session-details}Conclusão do Agendamento}
    \begin{center}
        \includegraphics[width=0.45\textwidth]{Imagens/sessions-06.png}
    \end{center}
    \legend{O Autor (2025)}
\end{figure}

Um ponto crucial é a segurança dos dados: todas as anotações salvas são criptografadas antes de serem persistidas no banco de dados (Figura \ref{fig:cripto-notes}), garantindo o sigilo das informações sensíveis dos estudantes.

\begin{figure}[H]
    \caption{\label{fig:cripto-notes}Dados do Prontuário Eletrônico Criptografados}
    \begin{center}
        \includegraphics[width=0.7\textwidth]{Imagens/sessions-03.png}
    \end{center}
    \legend{O Autor (2025)}
\end{figure}

\subsection{Área do Professor}

Os professores têm acesso a uma área específica onde podem visualizar as disciplinas que lecionam e a lista de alunos matriculados (Figura \ref{fig:professor}).

\begin{figure}[H]
    \caption{\label{fig:professor}Painel do Professor - Lista de Alunos por Disciplina Ofertada}
    \begin{center}
        \includegraphics[width=0.9\textwidth]{Imagens/professors-01.png}
    \end{center}
    \legend{O Autor (2025)}
\end{figure}

Ao acessar os detalhes de um aluno, o professor pode visualizar observações compartilhadas pelos profissionais de apoio (Figura \ref{fig:student-details}). Esta funcionalidade promove a integração entre o apoio psicopedagógico e o corpo docente, permitindo um acompanhamento mais holístico do estudante.

\begin{figure}[H]
    \caption{\label{fig:student-details}Detalhes do Aluno e Observações}
    \begin{center}
        \includegraphics[width=0.9\textwidth]{Imagens/professors-02.png}
    \end{center}
    \legend{O Autor (2025)}
\end{figure}

\section{Discussão dos Resultados}

A implementação do sistema traz melhorias significativas em relação ao processo manual anterior. A seguir, discutem-se os principais ganhos observados.

\subsection{Eficiência Operacional}

A automação do agendamento reduz drasticamente o tempo gasto pela profissional com a gestão de horários. Anteriormente, a confirmação de um único atendimento poderia exigir a troca de múltiplos emails e encontros presenciais. Com o sistema, o estudante visualiza a disponibilidade real e solicita o horário desejado, restando ao profissional apenas a aprovação com um clique.

\subsection{Segurança da Informação}

A substituição de anotações em papel ou arquivos de texto simples por um prontuário eletrônico criptografado elevou o nível de segurança. O controle de acesso baseado em perfis garante que apenas pessoas autorizadas visualizem dados sensíveis.

\subsection{Integração e Acompanhamento}

A funcionalidade de compartilhamento de observações com professores criou um canal formal e seguro de comunicação. Isso facilita a identificação precoce de dificuldades acadêmicas e permite que os professores adaptem suas estratégias pedagógicas baseados em informações concretas fornecidas pelo apoio psicopedagógico.

Em suma, o sistema atingiu os objetivos propostos, entregando uma solução robusta, segura e eficiente para o gerenciamento do apoio acadêmico no ICOMP.
