\chapter{Trabalhos Relacionados}
\label{chap:trabalhos}

Neste capítulo, são apresentados trabalhos acadêmicos e projetos de software que abordam temas correlatos ao deste TCC, como sistemas de agendamento online, gerenciamento de consultas pedagógicas e plataformas de apoio ao estudante. A análise destes trabalhos permite identificar funcionalidades consolidadas, tecnologias empregadas e, fundamentalmente, as lacunas que justificam o desenvolvimento da presente proposta.

\section{Sistema para Gerenciamento e Agendamento de Consultas para Psicólogos e Clientes}

\citeonline{souza2021} propôs uma plataforma web para a gestão de agendamentos de consultas pedagógicas. O sistema inclui funcionalidades essenciais como agendamento, registro de anamneses e geração de relatórios, operando com perfis de usuário distintos para o profissional e para o cliente. As tecnologias utilizadas foram PHP, JavaScript, HTML, CSS e MySQL.

A principal relevância deste trabalho reside na sua abordagem direta ao agendamento online no contexto da pedagogia, validando a arquitetura de perfis de usuário (profissional e aluno) que é similar à base do nosso sistema (profissional e aluno). No entanto, sua proposta é focada na relação dual cliente-profissional. O diferencial do nosso projeto se manifesta na introdução de um terceiro perfil, o de professor, e na criação de um ecossistema integrado ao ambiente acadêmico, com um canal de comunicação específico e controlado, funcionalidade não abordada por este trabalho.

\section{SystemPsi: Sistema Gerenciador para Psicólogos em Atuação Remota}

O SystemPsi é uma ferramenta tecnológica apresentada por \citeonline{stefen2022}, desenvolvida no Instituto Federal Catarinense (IFC). O sistema foi projetado para auxiliar profissionais em atendimentos remotos, e seu desenvolvimento contou com a validação de um profissional da área para garantir a pertinência das funcionalidades. A pilha de tecnologias inclui PHP e MySQL.

Este trabalho se destaca pela sua metodologia, que enfatiza a importância da validação das funcionalidades junto ao usuário final — o profissional de pedagogia. Essa abordagem reforça a metodologia adotada em nosso projeto, que também se baseia na colaboração direta com a profissional do ICOMP. O SystemPsi, contudo, concentra-se nas necessidades do profissional em um contexto de atuação remota geral. Nossa proposta avança ao especializar a ferramenta para o nicho universitário, atendendo não apenas às necessidades do profissional, mas também às dinâmicas de interação com alunos e professores dentro de uma instituição de ensino \cite{stefen2022}.

\section{Sistema para Agendamento de Serviços}

\citeonline{kieras2019} desenvolveu um sistema genérico para o agendamento de serviços, composto por uma aplicação web e um aplicativo móvel.

A principal contribuição deste trabalho para o nosso projeto não está no domínio da aplicação, mas sim na sua sólida documentação de engenharia de software. O detalhamento da análise de requisitos, dos casos de uso e do planejamento das funcionalidades serve como uma excelente referência metodológica para a estruturação do presente TCC. Enquanto o sistema da UTFPR foi projetado para ser genérico e aplicável a diversos contextos de agendamento, nosso sistema se aprofunda em um domínio específico — o apoio psicopedagógico universitário. Essa especialização permite a criação de funcionalidades sob medida, como o prontuário eletrônico e a visualização restrita para professores, que não fariam parte de uma solução genérica.
