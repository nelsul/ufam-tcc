\singlespacing

\noindent \CITACAOAUTOR. \textbf{\TITLE: \SUBTITLE. \ANO. \pageref{LastPage}f.} Undergraduate thesis (Computer Science) - Federal University of Amazonas, \LOCAL, \ANOD.

\vspace{0.5 cm}

\onehalfspacing

\noindent The growing demand for mental health support in higher education drives the search for technological tools that optimize service management. Aiming to enhance the psychopedagogical services offered at the Federal University of Amazonas (UFAM), this work presents the development of IcompCare, an Academic Support System. The web platform is designed to add efficiency, security, and organization to the process. The solution introduces an online scheduling portal where students can view staff availability. For professionals, the system offers a management dashboard and an electronic record module with encrypted data. The system also allows for the secure sharing of specific observations with professors, promoting integrated student follow-up. The project was developed using .NET 8, Vue.js, and PostgreSQL.

\noindent \textbf{Keywords:} psychological support, web system, academic management, data privacy, software engineering.

%\noindent Efficient diagnosis of emissions from combustion processes plays a key role in their control, an essential part of the overall effort to mitigate the increasing greenhouse effect. In industrial furnaces, a set of sensors ($CO_x$, $SO_x$, $NO_x$) at the exhaust is used to monitor pollutant rates, thus providing the necessary information for control purposes. In the case of natural gas furnaces, measurements of $O_2$ and $CO_2$ contents are used to check the condition of the combustion process. In this work, we develop a soft sensor to estimate the $O_2$ and $CO_2$ contents at the exhaust of a natural gas prototype furnace from images of flames grabbed by a charge-coupled device (CCD) camera. Feature vectors obtained from computer processing of the grabbed images are used as input data to identify autoregressive moving average (ARMAX) ``black box'' models having $CO_2$ content as output. Estimates of $O_2$ content by a Kalman filter running a preliminary ARMAX model helps the overall performance of the soft sensor. Results show that the flame dynamics identified model is capable of yielding statistically significant estimates of both $O_2$ and $CO_2$ composition in the flue gas up to $10\,s$ before the arrival of actual $O_2$ measurements. This outcome suggests that the inclusion of the proposed soft sensor in the closed-loop control strategy of similar combustion processes might be advantageous.