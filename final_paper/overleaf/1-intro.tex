\chapter{Introdução}
\label{cap:intro}

A gestão de serviços de apoio psicopedagógico em universidades como a UFAM demanda ferramentas eficientes. Processos manuais para agendamento, registro e comunicação são frequentemente lentos e apresentam riscos à privacidade dos dados dos estudantes. A tecnologia web oferece uma solução direta para modernizar e proteger esses fluxos de trabalho.

Este trabalho apresenta o desenvolvimento do IcompCare, uma plataforma web criada para centralizar e otimizar a gestão dos atendimentos no Instituto de Computação (ICOMP).

\section{Objetivos}

\subsection{Objetivo Geral}
Desenvolver um sistema web seguro e eficiente para automatizar o gerenciamento do serviço de apoio psicopedagógico do ICOMP/UFAM.

\subsection{Objetivos Específicos}
\begin{itemize}
    \item Levantar os requisitos funcionais e não funcionais do sistema.
    \item Projetar o banco de dados para o armazenamento seguro das informações.
    \item Implementar módulos de agendamento, prontuário eletrônico e perfis de acesso.
    \item Assegurar segurança através de criptografia.
\end{itemize}

\section{Metodologia}
A metodologia adotada foi baseada em um modelo de desenvolvimento iterativo e incremental, permitindo a construção do software em ciclos e validações contínuas com a stakeholder (psicóloga do ICOMP). Também foram adotadas práticas de engenharia de software como análise, projeto, desenvolvimento e teste.

\section{Organização do Trabalho}
O trabalho está organizado em seis capítulos. O Capítulo 2 apresenta o Referencial Teórico. O Capítulo 3 discute os Trabalhos Relacionados. O Capítulo 4 detalha Materiais e Métodos. O Capítulo 5 apresenta os Resultados e o Capítulo 6 as Considerações Finais. 

