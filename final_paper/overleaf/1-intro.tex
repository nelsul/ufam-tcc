\chapter{Introdução}
\label{cap:intro}

A gestão de serviços de apoio psicopedagógico em universidades como a UFAM demanda ferramentas eficientes. Processos manuais para agendamento, registro e comunicação são frequentemente lentos e apresentam riscos à privacidade dos dados dos estudantes. A tecnologia web oferece uma solução direta para modernizar e proteger esses fluxos de trabalho.

Este trabalho apresenta o desenvolvimento do IcompCare, uma plataforma web criada para centralizar e otimizar a gestão dos atendimentos no Instituto de Computação (ICOMP).

\section{Objetivos}

\subsection{Objetivo Geral}
Desenvolver um sistema web seguro e eficiente para automatizar o gerenciamento do serviço de apoio psicopedagógico do ICOMP/UFAM.

\subsection{Objetivos Específicos}
\begin{itemize}
    \item Implementar módulos de agendamento, prontuário eletrônico e perfis de acesso.
    \item Assegurar segurança através de criptografia.
\end{itemize}

\section{Metodologia} A metodologia adotada fundamentou-se no modelo de desenvolvimento iterativo e incremental. Essa abordagem permitiu a construção do software em ciclos evolutivos, garantindo validações contínuas e alinhamento constante com a principal stakeholder, a psicopedagoga do Instituto de Computação (ICOMP). O processo seguiu práticas consolidadas da Engenharia de Software, abrangendo as etapas de análise, projeto, implementação, testes e implantação.

A etapa inicial consistiu na elicitação de requisitos através de reuniões com a stakeholder, visando compreender as necessidades do fluxo de atendimento atual. Com base nessas definições, procedeu-se à seleção das tecnologias e à modelagem do banco de dados relacional. A fase de desenvolvimento seguiu uma ordem lógica, priorizando a implementação das regras de negócio e APIs no \textit{Backend}, seguida pela construção das interfaces de usuário no \textit{Frontend}.

Ao final do primeiro ciclo de desenvolvimento, foi realizada uma nova validação com a stakeholder. Esse feedback permitiu identificar melhorias e novos requisitos, que foram incorporados em iterações subsequentes. Por fim, estabeleceu-se um fluxo de entrega contínua para o \textit{deploy} automático e a validação do sistema em ambiente de produção.

\section{Organização do Trabalho}
O trabalho está organizado em seis capítulos. O Capítulo 2 apresenta o Referencial Teórico. O Capítulo 3 discute os Trabalhos Relacionados. O Capítulo 4 detalha Materiais e Métodos. O Capítulo 5 apresenta os Resultados e o Capítulo 6 as Considerações Finais. 

