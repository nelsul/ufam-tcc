%%%%%%%%%%%%%%%%%%%%%%%%%%%%%%%%%%%%%%%%%%%%%%%%%%%%%%%%%%%%%%%%%%%%%%%%%%%
%%%%%%%%%%%%%%%            FOLHA DE ROSTO                %%%%%%%%%%%%%%%%%%
%%%%%%%%%%%%%%%%%%%%%%%%%%%%%%%%%%%%%%%%%%%%%%%%%%%%%%%%%%%%%%%%%%%%%%%%%%%

\thispagestyle{empty}
\begin{titlepage}
\begin{center}
%%%%%%%%%%%%%%%%%%%%%
\MakeUppercase{\AUTOR}\\ %<<<<<<< NOME DO ALUNO
%%%%%%%%%%%%%%%%%%%%%%%%%%%%%%%%%%
\vspace{2.5cm}
%%%%%%%%%%%%%%%%%%%%
\large{\textbf{\TITULO}}\\ %<<<<<<< TÍTULO
%%%%%%%%%%%%%%%%%%%%%%%%%%%%%%%%%%
\vspace{2cm}
%%%%%%%%%%%%%%%%%%%%%
\normalsize{\textbf{Versão \VERSAO}} % Ou versão corrigida
%%%%%%%%%%%%%%%%%%%%%%%%%%%%%%%%%%
\vspace{2cm}
\end{center}

%%%%%%%%%%%%%%%%%%
%\hspace{.2\textwidth} % posicionando a minipage
\begin{flushright}
\begin{minipage}{0.6\textwidth}
{\noindent Trabalho de Conclusão de Curso apresentado ao Instituto de Computação da Universidade Federal do Amazonas como parte dos requisitos necessários para obtenção do grau de Bacharel em Ciência da Computação.} % Engenheira Mecânica}.
%{\noindent Dissertação apresentada à Universidade Federal do Amazonas, no Programa de Pós-Graduação em Engenharia Mecânica, para a obtenção do título de Mestre em Ciências.}
%{\noindent Tese apresentada à Universidade Federal do Amazonas, no Programa de Pós-Graduação em Engenharia Mecânica, para a obtenção do título de Doutor em Ciências.}
\\ \\
\end{minipage}
\end{flushright}
%%%%%%%%%%%%%%%%%%5

%\hspace{.2\textwidth} % posicionando a minipage
\begin{minipage}{0.69\textwidth}
%{Área de Concentração: Controle e Automação Mecânica} % para o caso de Dissertação/Tese
\end{minipage}
\\

\hspace{.2\textwidth} % posicionando a minipage
\begin{minipage}{0.69\textwidth}
%%%%%%%% ALTERE AQUI %%%%%%%%%%%%%
{Orientador: \ORIENTADOR}\\
%{Co-orientador: \COORIENTADOR}
%%%%%%%%%%%%%%%%%%%%%%%%%%%%%%%%%%
\end{minipage}
\vfill
\vspace{1.5cm}
\begin{center}
\large{\LOCAL\\\ANO}
\end{center}
\end{titlepage}
\pagebreak


%%%%%%%%%%%%%%%%%%%%%%%%%%%%%%%%%%%%%%%%%%%%%%%%%%%%%%%%%%%%%%%%%%%%%%%%%%%
%%%%%%%%%%%%%%%           FICHA CATALOGRÁFICA            %%%%%%%%%%%%%%%%%%
%%%%%%%%%%%%%%%%%%%%%%%%%%%%%%%%%%%%%%%%%%%%%%%%%%%%%%%%%%%%%%%%%%%%%%%%%%%
\pagenumbering{roman} \setcounter{page}{2}

% Para inserir a ficha catalográfica aqui em seu trabalho, você deve seguir os seguintes passos:
% 1) Você deve gerar sua ficha catalográfica no endereço http://fichacatalografica.ufam.edu.br/ficha/create e salvar o arquivo em formato .pdf;
% 2) Fazer o upload do arquivo na pasta "Imagens", do Overleaf ou da sua pasta LaTeX;
% 3) Por o endereço da localização do arquivo que você fez upload no comando \includepdf abaixo (não se esqueça de "descomentar" o comando, tirar o %).
% OBS: De preferência, não altere as formatações do offset deste comando.

                    % Comando para Ficha catalográfica

%\includepdf[pages=1,pagecommand={},offset=-2.5cm -3cm,angle=-90]{Imagens/emissaoO2CO2.pdf}




%%%%%%%%%%%%%%%%%%%%%%%%%%%%%%%%%%%%%%%%%%%%%%%%%%%%%%%%%%%%%%%%%%%%
\begin{center}
%%%%%%%%%%%%%%%%%
\textbf{\large{\TITULO}}
%%%%%%%%%%%%%%%%%%%%%%%%%%%%%%%%%%


\vspace*{1.5cm} % Para o Caso de mais de 3 membros na banca, mudar para {0.5cm}
%%%%%%%%%%%%%%%%%%%%
\normalsize{\MakeUppercase{\AUTOR}}
%%%%%%%%%%%%%%%%%%%%%%%%%%%%%%%%%%
\end{center}
\vspace*{1cm}% Para o Caso de mais de 3 membros na banca, mudar para {0.5cm}
{\noindent Trabalho de Conclusão de Curso (TCC) apresentado à Faculdade de Tecnologia da Universidade Federal do Amazonas como parte dos requisitos necessários para obtenção do grau de Engenheiro Mecânico.}
%{\noindent Trabalho de Conclusão de Curso apresentado à Faculdade de Tecnologia da Universidade Federal do Amazonas como parte dos requisitos necessários para obtenção do grau de Engenheira Mecânica.}
%{\noindent Dissertação apresentada à Universidade Federal do Amazonas, no Programa de Pós-Graduação em Engenharia Mecânica, para a obtenção do título de Mestre em Engenharia Mecânica.}
%{\noindent Tese apresentada à Universidade Federal do Amazonas, no Programa de Pós-Graduação em Engenharia Mecânica, para a obtenção do título de Doutor em Engenharia Mecânica.}
\\[0.25cm]
%{\noindent Área de concentração: Controle e Automação Mecânica.}\\[0.8cm] % Para dissertação/tese
%\vspace{-.8cm}


\noindent {Aprovado por:}
%\noindent {Aprovada por:} %no caso de tese/dissertação/monografia de pós

\vspace{.8cm}% Para o Caso de mais de 3 membros na banca, retirar isso
\begin{flushright}

{
\begin{center}

\rule{10cm}{.02cm} \\
%%%%%%%%%%%%%%%%%%%
{\AVALIADOR}\\ %Para o Caso de mais de 3 membros na banca, por igual ao que está no avaliador 4 e 5
{Orientador (\LOCALAVALIADOR)} \\
%%%%%%%%%%%%%%%%%%%%%%%%%%%%%%%%%
%\vspace{.45in}
\vspace{1.5cm}% Para o Caso de mais de 3 membros na banca, mudar para {.45in}
\rule{10cm}{.02cm} \\
%%%%%%%%%%%%%%%%%%%%%
{\AAVALIADOR}\\
{Membro (\LOCALAAVALIADOR)} \\
%%%%%%%%%%%%%%%%%%%%%%%%%%%%%%%%%%
%\vspace{.45in}
\vspace{1.5cm}% Para o Caso de mais de 3 membros na banca, mudar para {.45in}
\rule{10cm}{.02cm} \\
%%%%%%%%%%%%%%%%%%%%%
{\AAAVALIADOR}\\
{Membro (\LOCALAAAVALIADOR)}  \\
%%%%%%%%%%%%%%%%%%%%%%%%%%%%%%%%%%
\vspace{.45in}

%\rule{10cm}{.02cm} \\
%{\AAAAVALIADOR - Membro (\LOCALAAAAVALIADOR)}  \\
%\vspace{.45in}

%\rule{10cm}{.02cm} \\
%{\AAAAAVALIADOR - Membro (\LOCALAAAAAVALIADOR)}  \\
%\vspace{.45in}

\end{center}
}
\end{flushright}
\vspace{-.5cm}

\vfill
\begin{center}
%%%%%%%% ALTERE AQUI %%%%%%%%%%%%%
{\LOCAL, \DATA.}%MANAUS,AM - BRASIL \\
%%%%%%%%%%%%%%%%%%%%%%%%%%%%%%%%%%
\end{center}
\pagebreak


