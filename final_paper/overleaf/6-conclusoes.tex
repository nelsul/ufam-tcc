\chapter{Conclusões}
\label{chap:conclusao}

O desenvolvimento do Sistema de Apoio Acadêmico para o Instituto de Computação (ICOMP) da UFAM representou um passo importante na modernização e eficiência dos processos de atendimento aos estudantes. Este trabalho alcançou seu objetivo principal de projetar e implementar uma solução web capaz de gerenciar agendamentos, prontuários e o acompanhamento acadêmico de forma segura e integrada.

A adoção de tecnologias modernas como .NET 8 \cite{aspnetcore} e Vue.js 3 \cite{vuejs} permitiu a construção de uma aplicação robusta, escalável e com excelente experiência de usuário. A arquitetura limpa (Clean Architecture) utilizada no backend garante que o sistema seja de fácil manutenção e evolução, permitindo a incorporação de novas funcionalidades sem comprometer a estabilidade do código existente.

\section{Contribuições}

As principais contribuições deste trabalho podem ser resumidas em:

\begin{itemize}
    \item \textbf{Otimização do Fluxo de Trabalho}: A automação do agendamento eliminou gargalos operacionais, permitindo que a profissional de apoio dedique mais tempo ao atendimento dos alunos e menos a tarefas burocráticas.
    \item \textbf{Segurança e Privacidade}: A implementação de criptografia nos prontuários e o controle rigoroso de acesso asseguram a confidencialidade das informações, atendendo a requisitos legais e éticos.
    \item \textbf{Integração Pedagógica}: A ferramenta de comunicação entre o apoio psicopedagógico e os professores fortalece a rede de suporte ao estudante, permitindo ações mais coordenadas e eficazes.
    \item \textbf{Registro Histórico}: A centralização dos dados permite a geração futura de relatórios e estatísticas que podem embasar decisões institucionais sobre políticas de assistência estudantil.
\end{itemize}

\section{Trabalhos Futuros}

Apesar de o sistema atender aos requisitos iniciais, o desenvolvimento de software é um processo contínuo. Como sugestões para trabalhos futuros e evolução da plataforma, destacam-se:

\begin{itemize}
    \item \textbf{Integração com Sistemas da UFAM}: Implementar integração com o sistema acadêmico oficial da universidade para importação automática de dados de alunos, disciplinas e matrículas, eliminando a necessidade de cadastro manual.
    \item \textbf{Módulo de Relatórios Avançados}: Desenvolver um dashboard com indicadores de desempenho (KPIs), como número de atendimentos por período, principais motivos de procura e taxas de evasão, auxiliando na gestão estratégica.
    \item \textbf{Notificações via WhatsApp}: Expandir o sistema de notificações para incluir envio de mensagens via WhatsApp, aumentando a taxa de visualização e confirmação por parte dos estudantes.
    \item \textbf{App Mobile}: Desenvolver um aplicativo móvel nativo para estudantes, facilitando ainda mais o acesso ao agendamento e acompanhamento de suas solicitações.
\end{itemize}

Conclui-se que o sistema desenvolvido possui grande potencial para impactar positivamente a comunidade acadêmica do ICOMP, servindo como modelo que pode ser expandido para outras unidades da universidade.
