% FAVOR, NÃO ALTERAR NADA AQUI (NO MÁX, INSERIR PACOTES)

%\usepackage[bookmarks,pdftex,a4paper,colorlinks=true,citecolor=black,urlcolor=blue,linkcolor=black,pdfpagemode=None]{hyperref}
\usepackage[bookmarks,colorlinks=true,citecolor=black,urlcolor=blue,linkcolor=black,bookmarksopen=true,pdfpagemode=UseNone]{hyperref}
%\usepackage[colorlinks=true,linkcolor=blue,urlcolor=black,bookmarksopen=true]{hyperref}
%\usepackage{booktabs}
%\usepackage[open,openlevel=1]{bookmark}
\usepackage[centertags]{amsmath}
\usepackage{amsfonts}
\usepackage{amssymb}
\usepackage{amsthm}
\usepackage[T1]{fontenc}
%\usepackage[latin1]{inputenc}
\usepackage[utf8]{inputenc}
\usepackage[brazil]{babel}
\usepackage[alf,bibjustif,abnt-emphasize=bf,abnt-repeated-author-omit=no,abnt-etal-text=it]{abntex2cite} %ênfase na bibliografia em negrito
% "abnt-etal-text=it" - alteração do "et.al." para itálico feita por rodrigo nascimento em 06/2022
%\usepackage[alf,bibjustif,abnt-emphasize=em,abnt-repeated-author-omit=yes]{abntex2cite}
\usepackage{url}
%\usepackage{winfonts}
\usepackage{txfonts}
\usepackage{cancel} %para usar as barras de corte (cancelamento) \cancel, \xcancel, \bcancel, \cancelto{}{}
\usepackage{rotating} %para rotacionar imagens - %inserido por rodrigo nascimento em 06/22
\usepackage{graphicx} % para inserir figuras
\usepackage{float} %inserido por gustavo, fixar tabela e figura no local pondo \begin{figure ou table}[H]
\usepackage{ctable} %para funcionar a tabela do artigo
\usepackage{multirow} %inserir célula mesclada (gustavo)
\usepackage{colortbl} % usar tabela colorida do cronograma (gustavo) 
\usepackage[version=4]{mhchem} %inserido por Rodrigo (06/22), utilizar formulas quimicas
%\usepackage{ulem}
%\fontfamily{arial}\selectfont
%\renewcommand{\rmdefault}{arial}
%\usepackage{ltablex}%
\usepackage{longtable} % Pacote para tabelas grandes, que passem de uma página. Uma dica: o \usepackage{longtable} deve vir antes do \usepackage{arydshln} para não dar erro (conflito). Só resolveu quando pus nessa ordem - (gustavo)
\usepackage{arydshln} % por linhas pontilhadas dentro das matrizes (matriz em bloco) - (gustavo)
\usepackage{enumitem} % enumerar com letras. (inserido por gustavo)
\usepackage{subfig} % inserir subfiguras (gustavo)
\usepackage{setspace} % inserir espaçamento duplo, um e meio e simples (gustavo)
\usepackage{pdfpages} % inserir página em pdf inteira
\usepackage{ulem}
\usepackage{fancyhdr}
\usepackage{lastpage} %inserido por rodrigo nascimento 06/22 - contar total de páginas

\newcommand{\newappendix}{%
  \refstepcounter{chapter}\chapter*{Apêndice \thechapter}%
  %\addcontentsline{toc}{chapter}{Apêndice \thechapter}% Estava pondo em dobro no sumário
} % Tive que por isso pro apêndice funcionar (gustavo)

% Math -------------------------------------------------------------------
\newtheorem{theorem}{Teorema}{\bfseries}{\itshape}
\newtheorem{lemma}{Lema}{\bfseries}{\itshape}
\pagenumbering{roman} \setcounter{page}{1} % numeração em romanos, fixar como página 1
\newtheorem{definition}{Definição}{\bfseries}{\itshape}
\newtheorem{corollary}{Corolário}{\bfseries}{\itshape}
\newtheoremstyle{example}{\topsep}{\topsep}%
	{}%         Body font
	{}%         Indent amount (empty = no indent, \parindent = para indent)
	{\bfseries}% Thm head font
	{:}%        Punctuation after thm head
	{.5em}%     Space after thm head (\newline = linebreak)
	{\thmname{#1}\thmnumber{ #2}\thmnote{ #3}}%         Thm head spec
\theoremstyle{example}
\newtheorem{example}{Exemplo}
\newtheorem{proposition}{Proposição}
\newtheorem{algorithm}{Algoritmo}{\bfseries}{\itshape} %inserido por gustavo

\newcommand{\gbold}[1]{\mbox{\boldmath $\bf#1$}} %macro para letras gregas em negrito 
\newcommand{\lbold}[1]{\mbox{\boldmath $#1$}} %macro para letras latinas em negrito
\newcommand{\legend}[1]{{\footnotesize Fonte}: {#1}} % macro para inserir fonte de onde foi encontrada a figura (inserido por gustavo) 18/06/21
%\newcommand{\legend}[1]{{\small Fonte}: \textbf{#1}} 
\newcommand{\forceindent}{\leavevmode{\parindent=1em\indent}} %macro para forçar indentação - inserido por Rodrigo em 06/22

\sloppy

\usepackage[framed,numbered,autolinebreaks,useliterate]{mcode} % Para inserir ambiente código (Algoritmo). Inserido por Kamilla Cerdeira em 02/2021
%obrigatoriamente deve ter no diretório o arquivo "mcode.sty"
\renewcommand{\lstlistingname}{Código}
%\renewcommand{\lstlistingname}{Algoritmo}
\lstset{numberbychapter=false}


